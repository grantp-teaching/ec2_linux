\documentclass{pgnotes}

\title{EC2 Linux Lab}

\begin{document}

\maketitle

This lab exercise builds core skills for creating VPCs and EC2 instances.

\section{VPC creation} 

\subsection{VPC design}

\begin{enumerate}

\item
  Draw out a diagram for a VPC named \texttt{LAB\_VPC} using the 10.0.0.0/16 CIDR block with one subnet using the 10.0.1.0/24 CIDR block named \texttt{LAB\_1\_SN} and an internet gateway named \texttt{LAB\_IGW}.

\item 
  Write down (in words) the two rules that should govern the network routing.

\end{enumerate}

\subsection{VPC creation using console} 

\begin{enumerate}

\item
  Use the web console to create this VPC in AWS.
  Use the PowerShell script \url{check_lab_vpc.ps1} to check your work.

\item 
  Delete the VPC.

\end{enumerate}


\subsection{VPC creation using CLI} 

\begin{enumerate}

\item
  Use the AWS CLI (in PowerShell or Bash) to manually create the VPC using copy/paste of the IDs.

\item
    Use the PowerShell script \url{check_lab_vpc.ps1} to check your work.
  
\end{enumerate}

\section{EC2 setup}

Assuming your \texttt{LAB\_VPC} is setup already: 

\begin{enumerate}
\item Create a security group named \texttt{LAB\_SG} that allows SSH traffic inbound, and permits all traffic outbound.
\item Upload your private key to AWS (if not already there). 
\item Create an EC2 instance using Amazon Linux with the \texttt{t2.nano} type:
  \begin{enumerate}
  \item Look up the AMI ID automatically.
  \item Attach your security group and key pair to it.
  \item The default instance storage is fine.
  \end{enumerate} 
\item When the instance has started running, look at the screenshot and confirm its sitting at the login screen.
\item Use the ssh command in PowerShell / bash to connect to it. Use \texttt{ec2-user} and your private key as credentials. You will be at a standard bash prompt.
\item Apply system updates as suggested in the prompt.
\end{enumerate}

\subsection{EC2 termination}

Terminate your EC2 instance using the AWS CLI. 


\section{Automated setup} 

\begin{enumerate} 
  
\item
  Write a script in PowerShell (or Bash) to:
  \begin{itemize}
  \item exit immediately if a VPC named/tagged \texttt{LAB\_VPC} already exists.
  \item setup a VPC named/tagged \texttt{LAB\_VPC} using the CIDR block given.
  \item create one subnet named/tagged \texttt{LAB\_1\_SN} using the CIDR block given.
  \item create an internet gateway and attach it to the VPC.
  \item route all traffic to addresses outside of the VPC through the internet gateway
  \end{itemize}

\item
  Write a script in Powershell (or Bash) to remove the \texttt{LAB\_VPC} you built automatically.
  You will have to remove dependent components first.
  Suggested steps:
  \begin{enumerate}
  \item Get the VPC id corresponding to the \texttt{LAB\_VPC} by parsing the JSON from the \texttt{describe-vpcs} command.
  \item Get the internet gateway ID and delete it.    
  \item Get the subnet ids within this VPC (there will only be one here). Make sure to filter only the relevant subnets - you may have multiple VPCs later!
  \end{enumerate}
\end{enumerate}

\end{document}
